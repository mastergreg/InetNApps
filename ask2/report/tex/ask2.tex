%{{{ preamble
\documentclass[a4paper,9pt]{article}
\usepackage{anysize}
\marginsize{2cm}{2cm}{1cm}{1cm}
%\textwidth 6.0in \textheight = 664pt
\usepackage{xltxtra}
\usepackage{xunicode}
\usepackage{graphicx}
\usepackage{color}
\usepackage{xgreek}
\usepackage{fancyvrb}
\usepackage{minted}
\usepackage{listings}
\usepackage{enumitem} 
\usepackage{framed} 
\usepackage{relsize}
\usepackage{float} 
\usepackage{pstricks}
\usepackage{pst-node}
\usepackage{pst-blur}
\setmainfont[Mapping=tex-text]{FreeSerif}
%}}}
\begin{document}

\def\thesection {Μέρος \alph{section}}
\def\thesubsection {\roman{subsection}.}

\include{title/title}


\section*{Κώδικας άσκησης}
\subsection{Κεντρική σελίδα}
Εδώ μας δίνεται η δυνατότητα να επιλέξουμε το τι επιθυμούμε να δείξουμε καθώς
και δύο στοιχεία μορφοποίησης. Στοίχιση και χρώμα παρασκηνίου. Αυτά στέλνονται
με POST στο servlet XMLTransformer.
\inputminted[linenos,fontsize=\scriptsize]{html}{files/index.html}
\subsection{Servlet και web.xml}
Το servlet παίρνει από το request τα δεδομένα που θέλουμε και ανοίγει τα
αντίστοιχα αρχεία για να μας παρουσιάσει αυτά που ζητήσαμε.
\inputminted[linenos,fontsize=\scriptsize]{java}{files/XMLTransformer.java}
\inputminted[linenos,fontsize=\scriptsize]{xml}{files/web.xml}

\subsection{XSL αρχείο template}
Το template περιέχει την πληροφορία για τον τρόπο που θα περαστούν τα
δεδομένα, που βρίσκονται στα XML αρχεία, στη σελίδα, για να παρουσιαστούν στο
χρήστη.
\inputminted[linenos,fontsize=\scriptsize]{xml}{files/present.xsl}

\subsection{XML αρχεία που χρησιμοποιήθηκαν}
Όπως φαίνεται από τα παρακάτω δεν έχει σημασία το όνομα των πεδίων στα αρχεία
xml. Τα templates γράφτηκαν ώστε να λειτουργούν με τη μορφή και όχι με το
περιεχόμενό τους.
\inputminted[linenos,fontsize=\scriptsize]{xml}{files/Airplanes.xml}
\inputminted[linenos,fontsize=\scriptsize]{xml}{files/Books.xml}
\inputminted[linenos,fontsize=\scriptsize]{xml}{files/Cars.xml}


\end{document}
